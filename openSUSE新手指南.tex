%\documentclass[draft]{article}
\documentclass[10pt,openany]{book}
\usepackage{zpj}
\usepackage{ulem}
\usepackage{fancyvrb}
\newcommand{\command}[1]{\texttt{\textcolor{blue}{#1}}}
\newcommand{\soft}[1]{\texttt{\textcolor{blue}{#1}}}
\definecolor{codecolor}{RGB}{0,0,255}
\newcommand{\simp}[2]{\command{#1}$=$\command{#2}}
\begin{document}
%\zihao{-4}
\title{openSUSE新手指南\thanks{建议安装\command{okular}或\command{evince}来阅
读本文档。安装的命令参见第\ref{pre}节的zypper。本文档源文档现已
托管在\href{https://github.com/zpj-ustc/openSUSE-novice-guide}{Github}上
,欢迎对其进行修改分发。}}
%\author{朱沛俊\thanks{我的邮箱:\href{mailto:zpj.ustc@gmail.com}{zpj.ustc@gmail.com}}}
\author{\href{mailto:zpj.ustc@gmail.com}{\texttt{zpj.ustc}}}
\maketitle
\tableofcontents
\newpage
\chapter{基础部分}
让你折腾完了你就得到了一个勉强能用的openSUSE
\section{安装openSUSE}
安装openSUSE请参见\href{https://zh.opensuse.org/%E6%96%B0%E6%89%8B%E6%9D%91}{新手村}。
另外本文主要针对安装了KDE桌面环境的openSUSE 13.1的用户,其他环境如GNOME下可能会有某些地方不同。

对于桌面用户,一般只需要\texttt{/, /home, swap}三个分区即可,分区大小与具体使用有关。\texttt{swap}的大小一般桌面用户可以用公式$y=2\sqrt{x}$计算,其中$y$代表\texttt{swap}分区的大小,$x$代表内存大小,单位皆为GiB。而\texttt{/home}大小较为随意,最好不小于50G。根分区\texttt{/}则主要取决于你安装的程序(它们会在/usr中),比如你安装了\textsc{Matlab}就会占用7.9G,而Mathematica 会占用约4.5G,如果你将要安装这类大型软件,建议至少应该有60G的根分区空间。分区大小并没有严格的准则。
如果可能的话,为每个文件系统至少保留$25\%$的额外空间以应对今后的变化,还可以避免文件系统碎片。

\section{基本概念}
\subsection{Linux的一些基本概念}
\paragraph{终端} Shell是各种命令的入口,而终端相当于是为Shell提供了一个图形界面,如KDE下面的\soft{konsole}。\textcolor{red}{本文中大部分命令都是在终端中运行的。}

\paragraph{root权限} 顾名思义,root权限就是最根本权限。也就是说,有了root权限,你基本上有了做任何事的权力。
而执行某些命令,如安装、卸载软件包时需要root权限。此时你需要在命令前加sudo才能执行。
默认情况下输入密码时不会有类似于\texttt{****}的提示符。你也可以用su这个命令切换到root用户,
但是安全起见,在不必要的时候还是不要用su切换到root。

\paragraph{软件源} 软件源就是软件下载的来源。如果你已经添加的所有源里面都没有想要的软件包,但是在软
件包搜索或其他源中有,那你只需添加相应的源,再刷新源即可在zypper或YaST中找到。添加软件源的方法可以参见\href{https://zh.opensuse.org/SDB:%E6%B7%BB%E5%8A%A0%E8%BD%AF%E4%BB%B6%E6%BA%90}{添加软件源}
%\item[依赖关系]
\subsection[包管理器]{openSUSE下的包管理器}
openSUSE下安装软件一般先用zypper或YaST搜索,找到了便可以直接用zypper或YaST安装。
如果搜索不到,说明你添加的的所有软件源里面都没有包含这个软件,此时你就需要用软件包搜索了。搜到了便一键安装,否则便只能采取最后方法——自行编译了。如果你你有打包能力并且这个程序不违反OBS相关规定,那么你就把它丢到OBS上面去,
然后别人也可以用这个打好了的包了。
\paragraph{zypper}\label{pre} 它是一个命令行的包管理器,常用命令有
\begin{Verbatim}[formatcom=\color{codecolor}]
    zypper update #升级软件包
    zypper refresh #刷新软件源
    zypper install PKG1 PKG2 #安装PKG1和PKG2等软件包
    zypper remove PKG1 PKG2 #移除PKG1和PKG2等软件包
\end{Verbatim}
其中上述选项都可以简化输入,对应关系为:\begin{inparaitem}
 \item \simp{update}{up}
 \item \simp{refresh}{ref}
 \item \simp{install}{in}
 \item \simp{remove}{rm}
\end{inparaitem}
\paragraph{YaST} 这相当于Windows中的控制面板,让你无需打开命令行即可在图形界面完成包括软件包管理、用户和组管理等在内的各种任务。
\paragraph[软件包搜索与一键安装]{\href{http://software.opensuse.org/packages}{软件包搜索}与一键安装} 在这里你可以搜索你想要的软件包。
与zypper或YaST的区别在于可以找到你的源里面没有的软件。如果你利用\href{https://build.opensuse.org/}{OBS}编译了软件包,那么全世界的人都可以在这里使用它。
火狐浏览器自带了这个搜索引擎。在软件包搜索中找到软件以后,你就可以点击一键安装按钮了。一
键安装使得openSUSE中安装程序十分方便。在你进行软件包搜索后如果有多个源可以选择,同等条件下尽量选择
看起来比较官方的。\[\text{搜索软件}+\text{添加软件源}+\text{刷新软件源}+\text{安装软件}=\text{一条龙服务}\]
\subsection{KDE下启动程序的方法}
本文所提到的内容一般是任选下列两种方式中的某一种启动。
\paragraph{启动器} 全称Kickoff应用程序启动器,是类似于Windows中“开始菜单”的一个东西,一般可以用Alt+F1调出。新安装的程序可能暂时无法从这里启动。
\paragraph{krunner} 是类似于Windows中的“运行”的一个东西,一般可以用Alt+F2调出,之后输入你要运行或要搜索的程序即可。

\section{更新系统}
为了使你的更新过程尽可能块,你可能需要添加镜像源,比如中国科大镜像,可以参考相
应的\href{https://lug.ustc.edu.cn/wiki/mirrors/help/opensuse}{镜像使用帮助}之
后,按照提示添加并刷新软件源。

之后你就可以在终端中输入\command{sudo zypper up}更新系统了,这是很重要的一步。

运行zypper或YaST进行软件管理时,经常会由于系统后台正在运行一个zypper(系统只能同时运行一个zypper),
影响你添加软件源等各种操作。你可以在终端中输入\command{sudo kill pid}把这个进程杀掉,pid是它的进程号。

\section{本地化配置}
用LiveCD安装,无法得到完整的中文KDE环境,所以你需要安装
\begin{Verbatim}[formatcom=\color{codecolor}]
    kde4-l10n-zh_CN translation-update-zh_CN yast2-trans-zh_CN
\end{Verbatim}

为解决zip压缩包解压后文件名乱码的问题,请安装\soft{unzip-rcc}。

Fcitx是一个中文输入法框架,它对应的包为\soft{fcitx fcitx-config-kde4}。安装好后再启动它,从此你的openSUSE就可以输入中文了。
装好后它的默认词库还很弱,你可以安装一个比较大的词库(非必需)。可%
以参见\href{https://code.google.com/p/hslinuxextra/}{hs\-linux\-extra}以及%
\href{https://www.librehat.com/fcitx-sogou-pinyin-cell-database-convert-import-guide/}{词库教程}%
或者下载我编译好的\href{http://pan.baidu.com/s/1i3HtJ4T}{pyphrase}文件,复制
到家目录的\command{.config/fcitx/pinyin/}中。

当然你也可以不使用默认的拼音输入法而使用Fcitx的谷歌拼音(\soft{goo\-gle\-pin\-yin}),
搜狗拼音(\soft{so\-gou\-pin\-yin})等输入法模块。对于大部分输入法你都可以使用云拼音(\soft{cloud\-pin\-yin})等模块。
安装好云拼音以后,只要在Fcitx的附加组件设置中配置好,你的输入法就可以利用云拼音输入了。注意,某些云拼音来源
可能会失效,如果云拼音无法使用可能就是因为这个原因。
\section[双显卡]{双显卡}
如果你是NVIDIA用户,并且你的显卡支持,请参见\href{https://zh.opensuse.org/SDB:Bumblebee}{SDB:Bumblebee},强烈建议加源安装。

如果你是AMD用户\sout{,据室友说可以用命令关掉独立显卡,请自行Google。}
\chapter{各种应用}
介绍Linux下面常用的各类程序,本章用到且的OBS中没有的源一般会写在\ref{repo}中。
\section{Linux下的QQ}
腾讯官方提供的QQ for Linux早就无法使用了。webQQ已经挂了,SmartQQ基本半残,只能发送文字——总之很多方法都已经不行了。
但是Chrome和Chromium目前可以运行某些安卓程序,经过测试手机QQ2011可以成功在上面运行。
具体可以参考\href{http://huodong.ustc.edu.cn/Crx}{apk2crx}。
\section{影音播放}
{\color{orange}首先请安装好\href{https://lug.ustc.edu.cn/sites/opensuse-guide/codecs.php}{解码器},否则某些格式将无法播放!}。执行完这一步你的Packman源
就应该添加好了,如果你认为这个源太慢,请尝
试Packman的\href{http://packman.links2linux.org/mirrors}{镜%
像源}如台湾的\href{http://ftp.twaren.net/Linux/Packman/}{twaren}。
\begin{compactdesc}
 \item[电影] 推荐\soft{smplayer},或者\soft{vlc}。
 \item[音乐] 推荐\soft{deepin-music},启用\href{https://forum.suse.org.cn/viewtopic.php?f=7&t=2530}{百度插件}后便能够下载歌曲了。十分方便,
当然你也可以使用默认的\soft{amarok},但是前者搜索歌词,下载歌曲等功能更为强大
\end{compactdesc}
\section{浏览器}
openSUSE下面默认装有Firefox浏览器,它可以通过安装附加组件来增强其功能。

Firefox:默认安装

Chrome:添加Chrome的源后再搜索\soft{chrome}后安装相应的包即可。

Chromium:添加Packman源后再:
\begin{verbatim}
    sudo zypper in chromium chromium-ffmpeg chromium-pepper-flash
\end{verbatim}
即可安装好Chromium
\section{图像处理}
点阵图:\soft{gimp}

矢量图:\soft{inkscape}
\section{办公程序}
\TeX :推荐Kile作为前端,这种情况下你只需要
\begin{Verbatim}[formatcom=\color{codecolor}]
    sudo zypper in kile texlive-ctex texlive-savesym
\end{Verbatim}
更详细的配置过程请参
考\LaTeX的\href{https://forum.suse.org.cn/viewtopic.php?f=6&t=2392&p=18750}{配置指南}。除此之外你还可
以用\TeX studio, \TeX works等各种IDE,当然VIM和Emacs这类通用编辑器也是可以的,只不过可能比较难配置。本
文就是用\LaTeX排版的,怎么样,效果还不错吧。

Office:LibreOffice、永中Office或WPS。安装好WPS后需要下载安装\href{http://pan.baidu.com/s/1mgC3A0C}{Symbols}字体\footnote{感谢lzkxuan给我们提供此字体包,原帖地址:\url{https://forum.suse.org.cn/viewtopic.php?f=6&t=2697}}否则无法显示各种数学符号等。另外由于WPS是32位的程序,所以如果你是64位操作系统请额外安装\soft{cups-libs-32bit},否则WPS无法输出文档为pdf。

邮件客户端:\soft{thunderbird}, \soft{kmail}

文本编辑器:Kate, VIM, Emacs etc.
\section{教育程序}
词典:推荐\soft{GoldenDict}\footnote{请用软件包搜索安装版本号大于1.5的版本,否则无法读取.mdx格式},词典请上\href{http://pdawiki.com/forum/forum.php}{pdawiki}下载,
为了让你的词典更聪明,请下载\href{https://zpj.blog.ustc.edu.cn/wp-content/uploads/2014/02/wordsrule.tar.gz}{构词法}。

背单词:\soft{Parley}是一款很不错的背单词的程序%g,介绍可以参见

科学计算、数据处理:\soft{octave}、\textsc{Matlab}、Mathematica、\soft{maxima}、R(\soft{R-base})
\section{网盘}
\begin{compactitem}
 \item 坚果云\soft{nutstore}
 \item 百度云在openSUSE下有非官方的\soft{bcloud}客户端
 \item Dropbox
 \item Wuala
 \item etc.欢迎补充
\end{compactitem}
\section{下载程序}
\begin{compactdesc}
 \item[普通下载] KGet, Firefox的Down\-them\-all!插件,或者直接使用浏览器
 \item[BT] KTorrent, Deluge, Transmition
 \item[ed2k] aMule
\end{compactdesc}

大部分下载都可以用百度网盘这个神器离线下载好,然后利用有一份田网友的\href{http://git.oschina.net/youyifentian/dupanlink}{百度网盘助手}显示百度网盘文件的直接链接。(不过这个助手有时候可能会失效)
\chapter{更多}
\section{字体显示及其美化}
正式文档排版一般都需要宋体、黑体、仿宋、楷体这几种\href{http://pan.baidu.com/s/1mgiHWmO}{字体},下载好后点击即可安装。开源的字体如\soft{ubuntu-fonts wqy-microhei-fonts}等则可以直接在源中找到。

在字体设置中,调节各种字体字号为你喜欢的,个人认为屏幕显示字体最好选择无衬线字体,如Droid Sans Fallback、各种黑体。

为了使你的字体显示更加犀利,你需要将\soft{libfreetype6}相关的包的切换到opensuse\_zh源中,然后重新登录即可。可在YaST中进行操作。

\section{了解更多}
\begin{compactitem}
 \item \href{https://lug.ustc.edu.cn/sites/opensuse-guide/}{非官方指南}详细的介绍了一些简单的内容
 \item 有问题可以搜索\href{https://zh.opensuse.org/%E9%A6%96%E9%A1%B5}{openSUSE中文维基}
 \item \href{https://forum.suse.org.cn/}{openSUSE中文论坛}是openSUSE中文的官方论坛,坛主为MargueriteSu。
 \item 利用\href{https://google.com}{Google}搜索解决问题(如果无法打开,请尝试\href{https://code.google.com/p/goagent/}{Goagent})
 \item 自学Shell脚本语言,可以参考入门书籍TLCL:\href{http://home.ustc.edu.cn/~zpj/doc/Linux/The_Linux_Command_Line.pdf}{英文版}或\href{http://home.ustc.edu.cn/~zpj/doc/Linux/The_Linux_Command_Line(%e4%b8%ad%e6%96%87%e7%89%88).pdf}{中文版}
 \item 学习git,有电子书:\href{http://git-scm.com/book/}{英文版HTML}或中文版\href{http://git-scm.com/book/zh}{HTML}/\href{http://liam0205.me/attachment/Git/progit.zh.pdf}{pdf}
\end{compactitem}
\section{常见问题}
Konsole中可能没有采用恰当的字体,导致显示的光标与文字中间有很大的间隔,如果你发现你的Konsole有这个问题,
请依次进入Konsole的以下菜单:设置$\rightarrow$编辑当前方案$\rightarrow$外观$\rightarrow$选择
字体$\rightarrow$选中Droid sans Mono或其他等宽字体$\rightarrow$确定$\rightarrow$应用。

安装解码器等的时候,可能会遇到依赖关系导致的冲突的问题。
冲突的解决请参见\href{https://forum.suse.org.cn/viewtopic.php?t=2867&p=22491#p22491}{安装软件碰到碰到冲突,应该如何选择解决方式?}

更新时候常常会遇到以下情况:“将不会安装以下 51 个软件包的更新:blahblahblah”,若对此有疑问请
参见\href{https://forum.suse.org.cn/viewtopic.php?t=2777&p=21896}{帮助新手理解为什么有些包不会被更新}
\section{常用软件源}\label{repo}
%添加软件源可以用zypper/YaST。YaST中只需要以下步骤:
%YaST$\rightarrow$软件源$\rightarrow$添加$\rightarrow$下一步$\rightarrow$按照提示填写。


\begin{compactdesc}
 \item[\href{http://mirrors.hust.edu.cn/packman/suse/openSUSE_13.1/}{Packman}]
 提供多媒体编解码器、播放器、Broadcom无线网卡驱动、游戏等
 \item[\href{http://download.opensuse.org/repositories/home:/opensuse_zh/openSUSE_13.1/}{opensuse\_zh}]
 此软件源由中文用户维护,提供国内常用的软件。
 \item[\href{http://download.opensuse.org/repositories/KDE:/Extra/openSUSE_13.1/}{KDE:Extra}]
 含有大量额外的KDE程序。
 \item[\href{http://download.opensuse.org/repositories/GNOME:/Apps/openSUSE_13.1/}{GNOME:Apps}]
 含有大量GNOME程序。
 \item[\href{http://dl.google.com/linux/chrome/rpm/stable/i386}{Chrome~32位}|\href{http://dl.google.com/linux/chrome/rpm/stable/x86_64}{Chrome~64位}] 含有各种稳定版及非稳定版的Google Chrome浏览器
\end{compactdesc}
\section{KDE下的一些奇技淫巧}
\paragraph{快捷键} 在快捷键设置中你可以为程序或命令设置快捷键,这样你就可以抛弃桌面图标了——常用程序快捷键启动,非常用程序用启动器或Krunner启动。

\paragraph{特殊应用设置} 特殊应用程序设置参见\href{https://zpj.blog.ustc.edu.cn/?p=304}{我的博客}。

\paragraph{Dolphin中的缩略图} Dolphin可以开启缩略图功能,在配置菜单的常规中的预览中选中相应条目即可。如果没有你想要的条目比如pdf,那么你可以用thumbnail为关键字,搜索源中相应的软件包并安装即可。比如:
\begin{compactenum}
 \item \soft{kffmpegthumbnailer} 生成视频文件缩略图
 \item \soft{kdegraphics-thumbnailers} 生成pdf等文件缩略图
 \item \soft{kde-thumbnailer-epub} 生成epub文件缩略图
\end{compactenum}


\paragraph{屏幕反色} 在桌面效果的全部效果中可以设置反色功能,之后只需一个快捷键切换到夜间模式。

\paragraph{嵌入式终端} Kate、Kile等编辑器可以嵌入一个迷你终端,十分方便。

\paragraph{拆分视图} Dolphin的拆分视图功能使得在不同文件夹中复制文件等操作更加便捷。

\paragraph{色温调节} \soft{redshift}及其\soft{plasmoid-redshift}挂件使你能够方便的调节屏幕色温,保护视力。
\end{document}
